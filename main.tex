\documentclass{article}
\usepackage[utf8]{inputenc}
\usepackage[norsk]{babel} 
\usepackage{listings}
\usepackage[sfdefault,light]{roboto}  %font
\usepackage[T1]{fontenc} %font
\usepackage{graphicx} %images
\usepackage[table,xcdraw]{xcolor} %color in tables
\usepackage{array} %tabular: allow multiple lines
\usepackage{longtable} %table that can go over multiple lines
\usepackage{amsfonts} %checkmark
\usepackage{float} %force position
\usepackage{booktabs}
\newcommand{\tabitem}{~~\llap{\textbullet}~~}

\usepackage{geometry} %size of paper
 \geometry{
 a4paper,
 total={170mm,257mm},
 left=25mm,
 right=25mm,
 top=25mm,
 bottom=25mm,
 }

\newcolumntype{L}[1]{>{\raggedright\let\newline\\\arraybackslash}m{#1}} %better looking multirow cells

\usepackage{listings}
\usepackage{color}

\definecolor{dkgreen}{rgb}{0,0.6,0}
\definecolor{gray}{rgb}{0.5,0.5,0.5}
\definecolor{mauve}{rgb}{0.58,0,0.82}

\lstset{frame=tb,
  language=Java,
  aboveskip=3mm,
  belowskip=3mm,
  showstringspaces=false,
  columns=flexible,
  basicstyle={\small\ttfamily},
  numbers=none,
  numberstyle=\tiny\color{gray},
  keywordstyle=\color{blue},
  commentstyle=\color{dkgreen},
  stringstyle=\color{mauve},
  breaklines=true,
  breakatwhitespace=true,
  tabsize=3
}

\begin{document}
\begin{center}
\vspace*{4.5cm}
\Huge{TDT4120}\\[2pc]
\vspace{-1.5cm}
\noindent\rule{11cm}{0.8pt}\\

\Huge{Algoritmer og datastrukturer}\\[1pc]
\vspace{3.5cm}

\Large{2016}\\
%\vspace{1.0cm}
%\includegraphics[scale=0.6]{images/ntnu.png}
\end{center}

\setlength{\parindent}{3.5em}
%\setlength{\parskip}{1em} % space between index lines
\renewcommand{\baselinestretch}{1.1} % space between lines

\thispagestyle{empty}
\null
\newpage
\setcounter{page}{1}
\pagenumbering{roman}

\tableofcontents
\newpage

\setcounter{page}{1}
\pagenumbering{arabic}

\section{Innleding}
\newenvironment{boxed}
    {\begin{center}
    \begin{tabular}{|p{0.9\textwidth}|}
    \hline\\
    }
    { 
    \\\\\hline
    \end{tabular} 
    \end{center}
    }
\subsection{Hva er algoritmer?}
En algoritme er en hvilket som helst tydelig definert fremgangsmåte for beregninger som kan ta en verdi (eller en mengde verdier) som input og gir en verdi (eller en mengde verdier) som output. Man kan også se på det som et verktøy som skal løse et definert beregningsproblem. Når man definerer problemet, må man beskrive hvilket forhold man ønsker mellom input og output, for eksempel: “Input: Et veikart over en by og to punkter. Output: Den korteste veien (målt i meter) mellom de to punktene.”
\\\\
En algoritme er en veldefinert prosedyre som tar en verdi eller mengde verdier som input og produserer en verdi eller mengde med verdier som output. Dette er en sekvens som transformerer input til output.
At en algoritme er in-place vil si at den opererer på input-dataen uten å måtte lage f.eks. nye arrays for å løse problemet.
\\\\
At en algoritme er stabil vil si at hvis du sorterer en liste med tall, vil alltid tallet i forekomsten som var først i den opprinnelige listen komme først i den sorterte listen.

\subsubsection{Instans}
Hver samling av input-verdier til et problem kalles en instans. For eksempel kan man i en instans av problemet over ha inputverdier som er et veikart over Trondheim, og de to punktene kan være to geografiske punkter som tilsvarer NTNU Gløshaugen og NTNU Dragvoll.

\subsubsection{Problem}
En oppgave generert på generell input. Et problem er en relasjon mellom input og output.

\subsubsection{Probleminstans}
Et problem med et bestemt input. 

\subsubsection{Iterasjon}
En gjennomkjøring av en gjentatt handling.


\newpage
\section{Datastrukturer}
\subsection{Lenkede lister}
\subsubsection{Kjøretider}
\subsection{Abstrakte datastrukturer}
\subsubsection{Kø}
\subsection{Stack}
\subsection{Heap}
\newpage
\section{Kjøretidsberegning}
\subsection{Om kjøretider}
For å vite hvor effektiv en algoritme er, ønsker man å vite hvor kjapp den er i forhold til hvor mye informasjon man sender inn. Noen ganger vil en dobling av antall inn-verdier bare øke kjøretiden med en konstant tid, andre ganger vil den doble kjøretiden, og ofte vil den øke med veldig mye.
\\\\
Kjøretider skal alltid oppgis med \textbf{asymptotisk notasjon}. Det vil si at man finner (teoretisk) matematisk funksjon for kjøretiden med antall inn-verdier som parameter. Deretter ser man på det leddet som dominerer mest i uttrykket når inn-verdiene blir store nok, og bruker dette (uten noe konstantledd) som et mål på hvor rask algoritmen er. Målet er at det ikke skal vokse særlig raskt.

\begin{boxed}
Anta at du har mange forskjellige tall, og du skal finne det største taller. Du sjekker da det første tallet, og setter denne som en midlertidig mulighet for at det er det største tallet. Deretter går du gjennom resten av tallene og sammenligner hver og en av dem med det tallet du antar er det største. Hvis du plukker opp et større tall enn du har fra før, setter du det nye til å være en mulighet for å være det største. Når du har gått gjennom alle tallene, vil det tallet du satte av som en mulig kandidat sist være det største tallet. Slik må det være, fordi da du plukket opp det største tallet satte du dette som en mulig kandidat og ingen andre tall var større. Dette gjelder selvsagt også om flere tall er like i verdi. Så lenge du ikke kunne ane hvor i bunken det største er da du begynte, vet du at du er nødt til å gå gjennom alle tallene en gang.
\newline \newline
På en andre siden trenger du ikke gå gjennom tallene flere ganger heller, for ved å bruke algoritmen beskrevet ovenfor, vet du at svaret er korrekt etter nøyaktig en gjennomgang (så lenge du ikke gjør noen feil underveis, vel og merke!). Hvis mengden med tall dobles og du antar at du alltid vil bruke like lang tid på hver sammenligning uansett hvor mange tall du har, vil arbeidstiden også dobles. Det kan f.eks. se slik ut:
\newline \newline
Du har tallene 2 3 1 7 5
\newline \newline
1. steg: Du sjekker tallet 2; 2 er nå kandidat til å være det største tallet.\\
2. steg: Du sjekker tallet 3; 3 er nå kandidat til å være det største tallet.\\
3. steg: Du sjekker tallet 1; 3 er fortsatt størst.\\
4. steg: Du sjekker tallet 7; 7 er nå kandidat til å være det største tallet.\\
5. steg: Du sjekker tallet 5; 7 er fortsatt størst.
\newline \newline
Nå som alle tallene er sjekket vet vi at 7 er det største tallet.
\end{boxed}

\noindent Kjøretider kan man finne ved å bruke iterasjon, rekursjonstre, variabelskifte, substitusjonsmetoden eller masterteoremet.

\subsection{$\theta$, $O$ og $\Omega$-notasjon}
Disse symbolene brukes for å sammenligne hvor fort forskjellige funksjoner vokser, og det er viktig at man blir kjent med bruken av dem.

\subsubsection{$\theta$-notasjon}
Denne notasjonen er den mest nøyaktige av de tre. Har du denne har du også de to andre. Her er $f(n)$ en funksjon som beskriver kjøretiden til en algoritme, hvor $n$ er størrelsen på inputen. $g(n)$ er et annet funksjonsuttrykk som vanligvis er enklere enn $f(n)$. Nå har vi at $\theta(g(n))$ defineres som følger:

\begin{center}
\textit{$\theta(g(n)) = \{ f(n)$: slik at det finnes positive konstanter $c_1$, $c_2$ og $n_0$ så vi har $0 \leq c_1 g(n) \leq f(n) \leq c_2 g(n)$ for alle $n_0 \leq n \}$}
\end{center}

\noindent Dette vil si at $f(n)$ blir skvist mellom to kurver som begge bare har en konstant forskjell, så lenge input-verdien $n$ er stor nok. Litt unøyaktig forklart betyr det at $f(n)$ vokser "omtrent like raskt som" $g(n)$.

\begin{boxed}
Anta at du vet at kjøretiden til en algoritme er gitt nøyaktig til å være $g(n) = \frac{1}{2}\ n^2 + 3n$. Da vil denne sies å ha kjøretid $\theta(n^2)$ fordi ved å velge $c_1$ mindre enn \(\frac{1}{2}\) og $c_2$ større enn \(\frac{1}{2}\) kan vi finne en eller annen stor nok $n$ slik at $c_1 n^2$ alltid er mindre enn \(\frac{1}{2}\)$n^2 + 3$ og $c_2 n^2$ alltid vil være større enn \(\frac{1}{2}\)$n^2 + 3$.
\end{boxed}

\begin{table}[H]
    \caption{Eksempler}
    \label{tab:kjoretideks}
    \centering
    \begin{tabular}{|L{5em} | L{35em}|}
        \hline
        \rowcolor[HTML]{303F9F}
        \textbf{\textcolor{white}{Kjøretid}} & \textbf{\textcolor{white}{Beskrivelse}}\\
        \rowcolor[HTML]{E6E6E6}
        $\theta(lg n)$ & Denne kalles logaritmisk kjøretid, og er fin-fin. Dette skjer blant annet hvis du kan halvere problemstørrelsen din ved å teste ett element. F.eks. kan du tenke deg at du vet at du har en \textbf{stigende tallfølge} og skal finne et bestemt tall. Da kan du sjekke det midterste. Hvis det er for lite, kan du se bort fra alle tallene i venstre halvdel, som du vet er mindre. Dermed har du allerede omtrent halvert problemstørrelsen din! Om tallet du ser på først er større, ser du bort fra alle tallene i høyre halvdel, som er større enn dette. Dersom tallet du leter etter eksisterer i tallfølgen, finner du det fort. Og hvorfor er kjøretiden $\theta(lg n)$? Jo, du starter med en rekkefølge med lengde $n$ som du halverer gang på gang helt til du er nede i lengde 1. Hvis antallet halveringer som trengs er $k$, har vi \(\frac{n}{2^k}\)$ = 1$. Løsningen av denne ligningen er $k = lg n$. Merk også at $lg 2n = lg 2 + lg n = 1 + lg n$ – en dobling av problemstørrelse gir kun et konstant tillegg til kjøretiden. \\
        $\theta(n)$ & Dette er en polynomisk kjøretid. Hvis du har en input på størrelse $n$, og er nødt til å gå gjennom alle tallene én gang, har vi enkelt og greit $\theta(n)$.\\
        \rowcolor[HTML]{E6E6E6}
        $\theta(n^2)$ & Nok et eksempel på en polynomisk kjøretid. Gitt at du har en input på $n$. Det forekommer ofte at man må lete gjennom en matrise som har $n$ rader og $n$ kolonner. I dette tilfellet vil en dobling av input gi fire ganger så mange elementer å lete gjennom!\\
        $\theta(2^n)$ & Her er vi inne på eksponentiell kjøretid. Denne er ikke morsom! Bare ett lite tillegg på input fra f.eks. en million til en million og én vil øke kjøretiden med det dobbelte. Dette kan forekomme ved at du for hvert element du tester, springer det ut to nye valg som du må teste.\\
         \hline
    \end{tabular}
\end{table}

\subsubsection{$O$-notasjon}
Til forskjell fra $\theta$-notasjonen vil $O$-notasjonen kun ta for seg den øvre begrensningen til funksjonen:

\begin{center}
\textit{$O(g(n)) = \{ f(n)$: slik at det finnes positive konstanter $c$ og $n_0$ så vi har $0 \leq f(n) \leq cg(n)$ for alle $n_0 \leq n \}$}
\end{center}

\noindent Dette er ikke så ulikt $\thata$-notasjonen. $2n^2 + 100n$ er både $\theta(n^2)$ og $O(n^2)$. Men den er også $O(n^3)$, $O(n^4)$, $O(2^n)$ og alt som verre er. For hvis en eller annen konstant ganget med $n^2$ alltid vil være større enn kjøretiden, så gjelder det også alle andre funksjoner som vokser raskere enn det igjen. Altså, hvis du vet at kjøretiden til den kan også være $\theta(n*log(n))$ eller $\theta(n)$ eller $\theta(log(n))$. Derimot kan den ikke være verre, f.eks. $\theta(n^3)$.

\subsubsection{$\Omega$-notasjon}

\subsection{Noen vanlige kjøretider}
Noen vanlige kjøretider er beskrevet i tabellen under. Sortert fra høyest til lavest.

\begin{table}[H]
    \caption{Kjøretider}
    \label{tab:kjoretider}
    \centering
    \begin{tabular}{|L{15em} | L{15em}|L{15em}|}
        \hline
        \rowcolor[HTML]{303F9F}
        \textbf{\textcolor{white}{Kompleksitet}} & \textbf{\textcolor{white}{Navn}} & \textbf{\textcolor{white}{Type}}\\
        \rowcolor[HTML]{E6E6E6}
        $\theta(n!)$ & Factorial & Generell\\
        $\Omega(k^n)$ & Eksponensiell & Generell\\
        \rowcolor[HTML]{E6E6E6}
        $O(n^k)$ & Polynomsik & Generell\\
        $\theta(n^3)$ & Kubisk & Tilfelle av polynomisk\\
        \rowcolor[HTML]{E6E6E6}
        $\theta(n^2)$ & Kvadratisk & Tilfelle av polynomisk\\
        $\theta(n log n)$ & Loglineær & Kombinasjon av lineær og logaritmisk\\
        \rowcolor[HTML]{E6E6E6}
        $\theta(n)$ & Lineær & Generell\\
        $\theta(lg n)$ & Logaritmisk & Generell\\
         \rowcolor[HTML]{E6E6E6}
        $\theta(1)$ & Konstant & Generell\\
         \hline
    \end{tabular}
\end{table}

\begin{table}[H]
    \caption{Kjøretider}
    \label{tab:kjoretider}
    \centering
    \begin{tabular}{|L{8em} | L{8em}|L{8em}| L{8em}|L{8em}|}
        \hline
        \rowcolor[HTML]{303F9F}
        \textbf{\textcolor{white}{Algoritme}} & \textbf{\textcolor{white}{Best-case}} & \textbf{\textcolor{white}{Average-case}} & \textbf{\textcolor{white}{Worst-case}} & \textbf{\textcolor{white}{Sammenligning}}\\
        \rowcolor[HTML]{E6E6E6}
        Bubblesort & $\theta(n)$ & $\theta(n^2)$ & $\theta(n^2)$ & Ja\\
        Insertion sort & $\theta(n)$ & $\theta(n^2)$ & $\theta(n^2)$ & Ja\\
        \rowcolor[HTML]{E6E6E6}
        Selection sort & $\theta(n^2)$ & $\theta(n^2)$ & $\theta(n^2)$ & Ja\\
        Heapsort & $O(n lg n)$ & $O(n lg n)$ & $O(n lg n)$ & Ja\\
        \rowcolor[HTML]{E6E6E6}
        Quicksort & $\theta(n lg n)$ & $\theta(n lg n)$ & $\theta(n^2)$ & Ja\\
        Merge sort & $\theta(n lg n)$ & $\theta(n lg n)$ & $\theta(n lg n)$ & Ja\\
        \rowcolor[HTML]{E6E6E6}
        Counting sort & $\theta(n + k)$ & $\theta(n + k)$ & $\theta(n + k)$ & Nei\\
         \hline
    \end{tabular}
\end{table}
\newpage
\section{Rekursjon}
\subsection{Masterteoremet}
\subsubsection{Eksempel}
\newpage
\section{Sortering og søking}

\subsection{Sortering}

\subsubsection{Stabilitet}

\subsubsection{Sammenligningsbaserte sorteringsalgoritmer}

\noindent \textbf{Merge sort}\\
Merge-sort deler listen i to helt til den har en stor samling med lister som har bare ett element. Dermed har man en samling av sorterte lister. Disse slås så sammen ved å sammenligne det første elementet i to lister. Den minste legges fremst i en ny liste, og fjernes fra den opprinnelige listen. Dette gjøres for hver liste rekursivt, som til slutt gir en sortert liste med alle elementene.
\\\\
Merge-sort er en “splitt-og-hersk” algoritme. Merge-sort er effektiv. Deler opp problemet i stadig mindre biter, og når bitene er små nok flettes de sammen i sortert rekkefølge.

\begin{lstlisting}
    function MERGE-SORT(A,p,r)
    	if p < r then
    		q = $\left \lceil{(p + r)/2}\right \rceil$
    		MERGE-SORT(A,p,r)
    		MERGE-SORT(A,q + 1,r)
    		MERGE(A,p,q,r)
    	end if
    end function
\end{lstlisting}

\begin{lstlisting}
    function MERGE(A,p,q,r)
	    n1 = q - pr + 1
    	n2 = r - q
	    let L[1...n1 + 1] and R[1...n2 + 1] be new arrays
        for i = 1 to n1 do
	        L[i] = A[p + i - 1]
        end for
        for j = 1 to n2 do
        	R[j] = A[q + j]
        end for
        L[[n1] + 1] = $\infty$
        r[[n2] + 1] = $\infty$
        i = 1
        j = 1
        for k = p to r do
        	if L[i] $\leq$ R[j] then
        		A[k] = L[i]
        		i = i + 1
        	else
        		if A[k] = R[j] then
        			j = j + 1
        		end if
        	end if
        end for
        end function
\end{lstlisting}

\noindent \textbf{Eksempel}\\
Gitt tallfølgen 6, 5, 3, 1, 8, 7, 2, 4. Tallene deles opp i to grupper, 6, 5, 3, 1 og 8, 7, 2, 4. Disse deles så opp i to igjen; 6, 5 og 3, 1 og 8, 7 og 2, 4. Disse deles så opp i ett og ett tall. Sorteres så etter rekkefølge innad i toergruppene. Har da; 5, 6 og 1, 3 og 7, 8 og 2, 4. Slår så sammen to og to grupper til to grupper på fire, sortert: 1, 3, 5, 6 og 2, 4, 7, 8. Slår til slutt sammen de to firergruppene, sortert: 1, 2, 3, 4, 5, 6, 7, 8.

\begin{figure}[H]
\includegraphics[scale=0.7]{images/mergesort}
\centering %centering the image
\caption{Merge sort}
\label{fig:mergesort}
\end{figure}

\noindent \textbf{Quicksort}\\
Quicksort er enda en “splitt-og-hersk”-algoritme. Man starter gjerne Quicksort ved å randomisere listen. Den starter med å velge et pivotelement. Den deler deretter listen i to partisjoner: en med elementene som er mindre eller lik pivoten, og en med elementene som er større enn pivot. Deretter kaller den seg selv rekursivt på de to partisjonene. Deretter fletter man sammen de to partisjonene.

\begin{lstlisting}
    function QUICKSORT(A,p,r)
    	if p < r then
    		q = PARTITION(A,p,r)
    		QUICKSORT(A,p,q - 1)
    		QUICKSORT(A.q + 1, r)
    	end if
    end function
\end{lstlisting}

\begin{lstlisting}
    function PARTITION(A,p,r)
	    x = A[r]
    	i = p - 1
    	for j = p to r - 1 do
    		if A[j] $\leq$ x then
    			i = i + 1
    			exchange A[i] with A[j]
    		end if
    	end for
    	exchange A[i + 1] with A[j]
    	return i + 1
    end function
\end{lstlisting}

\noindent \textbf{Bubblesort}\\
Bubblesort tester to og to naboelementer. Dersom den første er større bytter de plass. Effektiv på små datamengder. Denne algoritmen bruker sammenligning.

\begin{lstlisting}
    for i = 0 to A.length - 1 do
    	for j = A.length downto i + 1 do
		    if A[j] < A[j - 1] then
		    	exchange A[j] with A[j - 1]
		    end if
	    end for
    end for
\end{lstlisting}

\noindent \textbf{Eksempel}\\
Gitt tallfølgen 6, 5, 3, 1, 8, 7, 2 4. Vil sorteres slik at etter en runde vil rekken bli 5, 3, 1, 6, 7, 2, 4, 8. Neste runde vil den være: 3, 1, 5, 6, 2, 4, 7, 8. Så vil den bli: 1, 3, 5, 2, 4, 6, 7, 8. Deretter: 1, 3, 2, 4, 5, 6, 7, 8. Etter fem runder blir rekkefølgen: 1, 2, 3, 4, 5, 6, 7, 8.\\

\noindent \textbf{Insertion sort}\\
Insertion-sort minner litt om selection-sort. Man tar det første usorterte elementet $x$, og sammenligner det med de sorterte elementene. Hvis man når et element som er mindre enn $x$ stopper man man gjennomgangen, og tar for seg neste $x$. Dette fungerer på samme måte som de fleste sorterer en korthånd.
\\\\
Insertion-sort er en enkel sorteringsalgoritme. Den tar de to første elementene og plasserer i forhold til hverandre. Deretter plasserer den det neste i forhold til de to forrige, osv. Veldig effektiv å bruke på små mengder. Kan f.eks. brukes i slutten på en Quick-sort algoritme.

\begin{lstlisting}
    for j = 0 to A.length do
	    key = A[j]
	    i = j - 1
	    while i > 0 and A[i] > key do
	    	A[i + 1] = A[1]
	    	i = i - 1
	    end while
	    A[i] + 1 = key
    end for
\end{lstlisting}

\noindent \textbf{Eksempel}\\
Gitt tallrekken 6, 5, 3, 1, 8, 7, 2, 4. Begynner med 6, som er første tall. Ser så på 5. 5 er mindre enn 6, og flyttes til plass null. Rekken ser da slik ut: 5, 6, 3, 1, 8, 7, 2, 4. Ser så på plass nummer to, tallet 3. Det er mindre enn både 5 og 6, og flyttes først. Tallrekken er da: 3, 5, 6, 1, 8, 7, 2, 4. Plass tre er 1. Det er minst av alle, og settes ført. Da er tallrekken: 1, 3, 5, 6, 8, 7, 2, 4. Plass nummer fire er tallet 8. Det er større enn alle tidligere tall, og blir stående. Plass fem er 7. Det er mindre enn 8 og flyttes til plass fire: 1, 3, 5, 6, 7, 8, 2, 4. Plass seks er 2, som er mindre enn nesten alle. Flyttes til plass en, og tallrekken blir: 1, 2, 3, 5, 6, 7, 8, 4. Da er det bare 4 igjen, som er mindre enn mange av tallene, og plasseres rett i plass tre. Da er tallfølgen ferdig sortert: 1, 2, 3, 4, 5, 6, 7, 8.\\

\noindent \textbf{Selection sort}\\

\subsubsection{Andre sorteringsalgoritmer}

\noindent \textbf{Heapsort}\\
Heapsort benytter seg av en heap som datastruktur. Lager en heap av alle elementene slik at hver node sine barn er mindre enn den selv. Det øverste elementet er alltid det største. Deretter plukkes det øverste elementet ut, for så å sortere heapen igjen, og ta ut det øverste elementet igjen. Fortsetter til heapen er tom.

\begin{lstlisting}
    BUILDMAXHEAP(A)
        for i = A.length downto 2 do
	        exchange A[1] with A[i]
	        A.heapSize = A.heapSize - 1
	        MAX-HEAPIFY(A,1)
        end for
\end{lstlisting}

\noindent \textbf{Counting sort}\\
Counting-sort tar et heltall $N$ mellom $0$ og $k$. Lager en liste med verdier fra $0$ til $k$ og setter inn tallene på sin plass i listen. Fungerer best når verdiene på tallene som sorteres ligger tett etterhverandre og $k$ ikke er for høy.

\begin{lstlisting}
    function COUNTINGSORT(A,B,k)
	    let C[0...k] be a new array
    	for i = 0 to k do
    		C[i] = 0
    	end for
    	for j = 1 to A.length do
    		C[A[j]] = C[A[j]] + 1
    	end for
    	for i = 0 to k do
    		C[i] = C[i] + C[i - 1]
    	end for
    	for j = A-length downto 1 do
    		B[C[A[j]]] = A[j]
    		C[A[j]] = C[A[j]] - 1
    	end for
    end function
\end{lstlisting}

\noindent Har kjøretid $O(n)$ i verste tilfelle, dersom $k = O(n)$ og $d = O(1)$.\\

\noindent \textbf{Eksempel}\\
Gitt A = [3, 3, 1, 4, 4, 3, 1, 2, 3, 5]. Hvordan ser C[0...6] ut idet Counting-Sort(A, B, 6) returnerer?
\\\\
[0, 0, 2, 3, 7, 9, 10] Tellesortering teller først forekomster, gjør så tellingene kumulative, og så dekrementerer disse under innsetting. Så tellingene til slutt tilsvarer altså hvor mange som er strengt mindre. F.eks. vil C[3] hvor mange verdier som er mindre enn 3. Her gis også enkelte andre svar noe uttelling. \\

\noindent \textbf{Radix sort}\\
Radix-sort sorterer tall i et gitt tallsystem (her titallsystemet) etter minste signifikante siffer.

\begin{lstlisting}
    function RADIXSORT(A,d)
    	for i = 1 to d do
    		Bruk Stabble sort til å sortere array A
    	end for
    end function
\end{lstlisting}

\noindent \textbf{Bucketsort}\\
Veldig lik Counting-sort, men bruker såkalte “bøtter” man putter tallene i. For eksempel [5,6) betyr at verdier større eller lik 5, men mindre enn 4 skal i bøtten.

\begin{lstlisting}
    function BUCKETSORT(A)
	    n = A.length
	    la B[0...n - 1] være et nytt array
	    for i = 0 to n - 1 do
	    	Gjør B[i] til en tom liste
	    end for
	    for i = 1 to n do
    		Sett inn A[i] i lista B[⌊nA[i]⌋]
    	end for
	    for i = 0 to n - 1 do
		    Sorter lista B[i] med INSERTIONSORT(B[i])
	    end for
	    Konkatiner listene B[0], B[1], …, B[n - 1] i rekkefølge
    end function
\end{lstlisting}

\noindent Bucket-Sort endrer kjøretid fra beste til verste tilfelle, og bruker ikke sammenligning.
\\\\
\noindent \textbf{Eksempel}\\
Lager bøtter for tallintervallene. Gitt tallene 29, 25, 3, 49, 9, 37, 21, 43. Lager bøtter for 0-9, her havner 3 og 9. 10-19 er tom. 20-29 år 21, 25 og 29. 30-39 får 37. 40-49 får 43 og 49.

\begin{figure}[H]
\includegraphics[scale=0.5]{images/bucketsort1}
\centering %centering the image
\caption{Bucketsort første trinn}
\label{fig:bucketsort1}
\end{figure}

\noindent Når dette så skal setter sammen til en sortert tallfølge hentes tallene opp fra bøttene igjen.

\begin{figure}[H]
\includegraphics[scale=0.5]{images/bucketsort2}
\centering %centering the image
\caption{Bucketsort andre trinn}
\label{fig:bucketsort2}
\end{figure}

\noindent Tallfølgen blir da til slutt 3,9, 21, 25, 29, 37, 43, 49.\\

\noindent \textbf{Topologisk sortering}\\
Topologisk sortering bruker en DAG til å finne en rekkefølge gjennom alle elementene i grafen. Det tillates ikke sykler. Man begynner i noden som ikke har noen kanter inn til seg.

\begin{lstlisting}
    function TOPOLOGICALSORT(G)
    	DFS(G) to compute finish times v.f for each vertex v.
    	as each vertex is finished, insert it onto the end of the list
    	return the list of vertices
    end function

\end{lstlisting}


\subsection{Søking}

\subsubsection{Brute force}

\subsubsection{Binærsøk}
Et binært søketre er et tre som tilfredsstiller binært-søketre-egenskapen.

\begin{table}[H]
    %\caption{}
    \label{tab:binaert}
    \centering
    \begin{tabular}{|L{10em} | L{10em}| L{10em}|}
        \hline
        \rowcolor[HTML]{303F9F}
        \textbf{\textcolor{white}{Operasjon}} & \textbf{\textcolor{white}{Average}} & \textbf{\textcolor{white}{Worst}}\\
        \rowcolor[HTML]{E6E6E6}
        Plass (bit) & $O(log n)$ & $O(n)$\\
        Søk & $O(log n)$ & $O(n)$ \\
        \rowcolor[HTML]{E6E6E6}
        Sett inn & $O(log n)$ & $O(n)$ \\
        Slett & $O(log n)$ & $O(n)$ \\
         \hline
    \end{tabular}
\end{table}

\noindent \textbf{Eksempel}\\
Hvis du setter verdiene 1, 2, 9, 5, 10, 7, 6, 4, 8 og 3 inn i et tomt binærtre (én etter én, i oppgitt rekkefølge), hva blir høyden til treet (antall kanter i den lengste stien fra rota til en løvnode)?

5
\\\\
\noindent \textbf{Eksempel}\\
La $x$ være en gitt node i søketreet. Hvis $y$ er en node i det venstre subtreet til $x$ må $y$ sin verdi være mindre eller lik ($\leq$) $x$ sin verdi. Tilsvarende for høyre subtre. Her må $y$ sin verdi være større enn eller lik ($\geq$) $x$ sin verdi.

\begin{figure}[H]
\includegraphics[scale=0.7]{images/binaer}
\centering %centering the image
\caption{Binært tre}
\label{fig:binaer}
\end{figure}


\newpage
\section{Grafer og grafalgoritmer}
Det er tre måter å traversere et binært tre på; preorder, inorder og postorder.
\begin{itemize}
    \item \textbf{Preorder}: Her printer man ut nodens verdi før dens barn, venstre og deretter høyre. 
    \begin{itemize}
        \item Eksempel fra figuren under: 10, 6, 3, 2, 1, 4, 5, 8, 7, 9, 13, 11, 12, 18, 15, 14, 16, 17
    \end{itemize}
    \item \textbf{Inorder}: Her printer man venstre barn, noden, og deretter høyre barn (om ikke det er noe venstre barn, print noden før høyre barn)
    \begin{itemize}
        \item Eksempel fra figuren under: 1, 2, 3, 4, 5, 6, 7, 8, 9, 10, 11, 12, 13, 14, 15, 16, 17, 18
    \end{itemize}
    \item \textbf{Postorder}: Her printer man nodens verdi etter man har printet venstre og høyre barn
    \begin{itemize}
        \item Eksempel fra figuren under: 1, 2, 5, 4, 3, 7, 9, 8, 6, 12, 11, 14, 17, 16, 15, 18, 13, 10
    \end{itemize}
\end{itemize}

\begin{figure}[H]
\includegraphics[scale=0.7]{images/tregraf}
\centering %centering the image
\caption{Tre}
\label{fig:tregraf}
\end{figure}

\subsection{Representasjon}
\subsubsection{Nabolister}
\subsubsection{Nabomatriser}
\subsection{Traversering}
\subsubsection{Bredde-først-søk (BFS)}
BFS implementeres med en kø. BFS utforsker grafen i bredden. Man starter på foreldrenoden og legger inn alle dens barn i køen. Når alle naboer til node $x$ er oppdaget, fjernes den fra køen og man tar den neste noden i køen og legger alle dens barn inn i køen. Når køen er tom, sjekker man ikke videre om det er ubesøkte noder.

\begin{lstlisting}
    function BFS(G,v) // v er startnode
	    lag en kø Q
    	legg v inn i Q
    	while Q.notEmpty() do
    		v = Q.dequeue()
    		for each edge e adjacent to v do
    			if e not marked then
    				mark w
    				Q.enqueue(e)
    			end if
    		end for
    	end while
    end function

\end{lstlisting}


\noindent \textbf{Traversering}

\subsubsection{Dybde-først-søk (DFS)}
DFS implementeres med en stack. DFS utforsker grafen i dybden. Den fyller stacken med noder den støter på. Kan man ikke gå videre vil den “backtrace” til forrige node og se etter en mulig vei videre. Hvis stacken er tom, sjekker man om alle noder er besøkt. Hvis ikke, starter man på nytt fra ubesøkte noder.

\begin{lstlisting}
    function DFS(G,v)	//v er startnode
	    initialiser en tom stack, S
    	for each vertex u in G do
    		set visited[u] $\rightarrow$ false
    	end for
    	S.push(v)
    	while S.notEmpty() do
    		u = S.pop()
    		for all w adjacent to u do
    			if not visited[w] then
    				visited[w] $\rightarrow$ true
    				S.push(w)
    			end if
    		end for
    	end while
    end function
\end{lstlisting}

\noindent \textbf{Traversering}

\subsection{Korteste vei}
\subsubsection{En til alle}
\textbf{Dijkstras algoritme}\\
Dijkstra er en korteste vei, en-til-alle-algoritme. Tillater ikke negative kanter. Den velger noder en etter en fra hvor nærme de er startnoden. Dijkstra er en grådig algoritme.

\begin{lstlisting}
    function DIJKSTRA(G,w,s)
	    INITIALIZE-SINGLE-SOURCE(G,s)
    	S = Ø
    	Q = G.V
    	while Q $\neq$ Ø do
    		u = EXTRACT-MIN(Q)
    		S = S $\cup$ {u}
    		for each vertex v $\in$ G.Adj[u] do
    			RELAX(u,v,w)
    		end for
    	end while
    end function
\end{lstlisting}

\textbf{Bellman-ford}\\
\textbf{DAG shortest path}\\
DAG-Shortest-Path er en korteste vei, en-til-alle algoritme. Tillater ikke negative kanter, og kan selvfølgelig ikke ha sykler, da det er en DAG. Gjør topologisk sortering av DAGen og besøker hver node en gang for å kjøre RELAX på nodene foran.

\begin{lstlisting}
    function DAG-SHORTEST-PATH(G,w,s)
    	TOPOLOGICAL-SORT(G)
    	INITIALIZE-SINGLE-SOURCE(G,s)
    	for each vertex u, taken in topologically sorted order do
    		for each vertex v ∈ G.Adj[u] do
    			RELAX(u,v,w)
    		end for
    	end for
    end function

\end{lstlisting}

\subsubsection{Alle til alle}
\textbf{Floyd-Warshall}\\
Floyd-Warshall er en korteste vei, alle-til-alle-algoritme. Den bruker DP. Lager en nabomatrise for alle noder og hvor det går kantvekter. Konstanden mellom disse blir verdien av kanten. Er det ikke en direkte vei mellom to noder settes verdien til $\infty$. Deretter velger den en node a og sjekker om veien fra u til v er kortere via a. Deretter finner den en ny node, og sjekker om det er kortere veier om man benytter seg av denne. Slik fortsetter den til den har besøkt alle noder.

\subsection{Maksimal flyt}
\subsubsection{Flytnettverk}
\subsubsection{Residualnettverk}
\subsubsection{Flytforøkende sti}
\subsubsection{Minimalt kutt}
Minimum-snitt (min-cut) på et flytnettverk: det snittet som har lavest kapasitet av alle snitt. Det vil si min-cut angir en flaskehals i flytnettverket. Det vil si at det ikke kan sendes mer flyt gjennom nettverket enn det vi kan sende gjennom flaskehalsen. Man kan da ikke finne noen flytforøkende sti over flaskehalsen. Det vil da være maksimal-flyt, max-flow.\\

\noindent \textbf{Ford-Fulkersons metode}\\
Ford-Fulkerson-metoden finner maksimal flyt i et flytnettverk. Hver iterasjon forsøker å finne en flytforøkende sti, og setter på all den flyten som er mulig. Deretter leter den etter en ny flytforøkende sti, og gjentar prosessen. Når det ikke er flere flytforøkende stier har man oppnådd maksimal flyt. Den benytter seg av DFS for å finne flytforøkende sti.
\\

\noindent \textbf{Edmonds-Karp}\\
Endret en bokstav på Ford-Fulkerson-metoden. De benytter seg av BFS. Edmonds-Karp bruker Ford-Fulkerson og BFS til å finne flytforøkende stier.\\
\newpage
\section{Dynamisk programmering}
\subsection{Lengste felles understreng}
\subsection{Rod-cutting}
\newpage
\section{Grådige algoritmer}
En grådig algoritme er en algoritme som velger den beste lokale løsningen. Det vi si at den velger en lokalt optimal løsning i håp om at den også skal være globalt optimal.
\\\\
Problemene vi bruker grådighetsalgoritmer på ligner ofte veldig på problemene som er beskrevet tidligere i dette kapittelet. Vi har fortsatt optimal substruktur og vi har valg. Forskjellen ligger i at valgene er mye enklere. Vi tviler ikke lenger på hva som er best, en ting skiller seg klart ut og vi kan velge det hver gang uten å trenge å vurdere de andre. Merk at det ikke alltid vil fungere å bruke en grådig algoritme.
\\\\
Problemer som kan løses med grådige algoritmer har ikke nødvendigvis overlappende delproblemer, men de har det som kalles \textit{greedy-choice property}. Det går ut på at et valg som er lokalt optimalt (dvs. at det ser ut som et godt valg her og nå) er den del av den globalt optimale løsningen (dvs. at det vil føre frem å ta det grådige valget).

\subsection{Huffmankode}
Huffmankode er en særdeles effektiv teknikk for å komprimere data. Ideen er å ha en variabel lengde på kodebitene. For eksempel i en enkel tekstfil lar man de bokstavene som forekommer ofte ha en kort kode, og de man bruker sjelden gir man en lang kode.
\\\\
En prefikskode er slik at intet kodeord også er et prefiks til et annet kodeord. Dette gjør dem meget lett å dekode, for man trenger da ikke vite hvor et ord slutter og et annet begynner. En dekodingsprosess trenger en hendig representasjon for prefiks-koden slik at det opprinnelige kodeordet lett blir funnet. En god representasjon er rett og slett et binærtre hva bladene er de gitte bokstavene. Huffman fant opp en grådighetsalgoritme som konstruerer en optimal prefikskode. Huffman-algoritmen bruker frekvensen til de forskjellige bokstavene og lager et binærtre.

\begin{boxed}
Anta at vi sender noe informasjon som kun innehar bokstavene e, r, s og t. Vi vet også den relative frekvensen til bokstavene. Dette er vist i tabellen:
\begin{table}[H]
    %\caption{}
    \label{tab:huffman1}
    \centering
    \begin{tabular}{|L{5em} |L{5em}|L{5em}|L{5em}|L{5em}|}
        \hline
        \rowcolor[HTML]{303F9F}
        \textbf{\textcolor{white}{Bokstav}} & \textbf{\textcolor{white}{e}} & \textbf{\textcolor{white}{r}} & \textbf{\textcolor{white}{s}} & \textbf{\textcolor{white}{t}}\\
        \rowcolor[HTML]{E6E6E6}
        Frekvens & 45 & 27 & 15 & 13\\
         \hline
    \end{tabular}
\end{table}
Nå bruker vi algoritmen til Huffman steg for steg og kommer til slutt fram til det optimale binærtreet for disse fore bokstavene. Dermed har vi koden til de forskjellige bokstavene, som blir:
\begin{table}[H]
    %\caption{}
    \label{tab:huffman1}
    \centering
    \begin{tabular}{|L{5em} |L{5em}|L{5em}|L{5em}|L{5em}|}
        \hline
        \rowcolor[HTML]{303F9F}
        \textbf{\textcolor{white}{Bokstav}} & \textbf{\textcolor{white}{e}} & \textbf{\textcolor{white}{r}} & \textbf{\textcolor{white}{s}} & \textbf{\textcolor{white}{t}}\\
        \rowcolor[HTML]{E6E6E6}
        Kode & 0 & 10 & 111 & 110\\
         \hline
    \end{tabular}
\end{table}
Legg merke til at dette er en prefikskode. Ordet ''se'' blir 1110, ordet ''erterester'' blir da 0101100100111110010.

\begin{figure}[H]
\includegraphics[scale=0.47]{images/huffman}
\centering %centering the image
\caption{Huffman-tre}
\label{fig:huffman}
\end{figure}
\end{boxed}

\subsection{Bevis ved fortrinn}
\subsection{Bevis ved forsprang}
\newpage
\section{Trær}
\begin{figure}[H]
\includegraphics[scale=0.5]{images/traer}
\centering %centering the image
\caption{Tre}
\label{fig:trær}
\end{figure}

\begin{itemize}
    \item \textbf{Nivå for node} er antall grener som må passeres f.o.m. rot t.o.m. noden. Merk at rotnoden er på nivå 0.
    \item \textbf{Nodegrad} er antall barn (som er det samme som antall subtrær/undertrær) en node har.
    \item \textbf{Fritt tre} er at grenene ikke har retning, så alle nodene kan oppfattes som rotnode.
    \item \textbf{Rettet tre} vil si at grenene har retning, som i Figur \ref{fig:trær}.
    \item \textbf{Ordnet tre} vil si hvordan subtrærne/barna er ordnet i forhol til hverandre. Hvis det er viktig at A er til venstre, D i midten og E til høyre, er det et ordnet tre. Hvis rekkefølgen ikke har noe å si er det et uordnet tre.
    \item \textbf{Løvnode} er er node uten barn.
    \item \textbf{Indre node} er en node med barn.
    \item \textbf{Trehøyde} er maks antall grener som kan passeres f.o.m. rot t.o.m. løvnode.
    \item \textbf{$k$-grad-tre} vil si et tre der hver node kan ha maks $k$ barn. Posisjonen til barna er viktige.
    \item \textbf{Fullt $k$-grad-tre} er et $k$-grad-tre der alle indre noder har $k$ barn.
    \item \textbf{Komplett $k$-grad-tre} vil si et $k$-grad-tre der alle indre noder har $k$ barn, og enhver løvnode ligger på nivå $h$ eller $h - 1$, hvor $h$ er dybden (altså enten nederst i treet eller nest nederst). Antall noder på dybde $h$ er $k^h$. Trehøyden er $log_k n$ der $n$ er antall løvnoder.
\end{itemize}

\subsection{Implementasjon}
Hvordan vi implementerer en trestruktur avhenger av om vi har fast antall barn (f.eks. binære trær) eller variabelt antall barn (f.eks. B-trær).
\\\\
For hver av variantene har vi tatt med kjøretiden til to vanlige operasjoner:
\begin{itemize}
    \item finne forelder til en node
    \item finne barn nummer \textit{i} til en node
\end{itemize}

\noindent I generelle trær med fast antall barn har hver node er et objekt med en verdi, en peker til hver av barna og en peker til far.
\\\\
I generelle trær med variabelt antall barn er det to alternativer. Alternativ 1 er at hver node er et objekt med en verdi, en peker til sitt første barn (lengst til venstre), en peker til far og en til sin nærmeste bror til høyre for seg. I alternativ 2 utnytter vi at alle noder i et tre (untatt roten) har nøyatig én far. Vi lager en array. På plass nr \textit{i} står faren til node nummer \textit{i}, eller en peker til denne. Det blir da lett å finne faren, men for å finne barna til node \textit{i}, må vi søke gjennom arrayen etter tallet \textit{i}.

\begin{figure}[H]
\includegraphics[scale=0.6]{images/fedrearray}
\centering %centering the image
\caption{Figuren viser et tre og den tilsvarende "fedrearrayen". Arrayen vil ha lke mange elementer som det er noder.}
\label{fig:fedrearray}
\end{figure}

\noindent Alternativ 3 er å lagre nodene i en array. HVert element i arrayen inneholder et objekt med verdien til noden og pekere til en lenket liste. Den lenkede listen inneholder pekere til barna.

\begin{figure}[H]
\includegraphics[scale=0.6]{images/alernativ3}
\centering %centering the image
\caption{Figuren viser alternativ 3.}
\label{fig:alternativ3}
\end{figure}

\subsection{Binære trær}
Binære trær er egentlig et 2-grad-tre, men dette navnet blir aldri brukt. Hver node har altså maks to barn og ordningen av høyre- og venstrebarn er viktig. Hver node er et objekt med en verdi, en peker til hver av barna og en peker til faren.

\begin{figure}[H]
\includegraphics[scale=0.6]{images/binaeretraer}
\centering %centering the image
\caption{Disse to trærne er forskjellige!}
\label{fig:binaeretraer}
\end{figure}

\subsubsection{Binære søketrær}
Et binært søketre er bygget som et binært tre, men det er i tillegg organisert slik at verdien til venstre barn er mindre eller lik verdien til far mens verdien til høyre barn er større. Det betyr at for alle noder gjelder at alle verdiene til nodene i venstre subtre er lavere og alle i høyre er høyere enn verdien til noden selv. 

\begin{figure}[H]
\includegraphics[scale=0.6]{images/binaeresoeketraer}
\centering %centering the image
\caption{Binært søketre}
\label{fig:binaeresoeketraer}
\end{figure}

\noindent Poenget er at med denne organiseringen er det vanligvis ganske kjapt å finne fram til ønsket node. Hvor lang tid det tar avhenger imidlertid av hvor heldig vi er med rekkefølgen nodene settes inn i. Hvor lang tid det tar å finne et element i verste tilfelle avhenger direkte av høyden til treet.
\\\\
\textbf{Innsetting av node}\\
Innsetting av noder i et binært tre er enkelt. Det er bare å passe på å overholde kriteriene for et binærtre. Vi starter alltid ved roten. Dersom treet ikke har noen rot, dvs. treet ikke er påbegynt, settes vår node til å være rotnoden. Ellers sjekker vi for hver nye node vi kommer til om verdien til noden vi skal sette inn er større eller mindre enn verdien til denne noden. Dersom den er mindre, går vi til venstre, ellers går vi til høyre. Vår node kan settes inn på første tomme plass vi kommer til.

\begin{figure}[H]
\includegraphics[scale=0.45]{images/innsettingnode}
\centering %centering the image
\caption{Sette inn node i binærtre}
\label{fig:innsettingnode}
\end{figure}

\noindent\textbf{Fjerning av node}\\
Vi har tre muligheter:
\begin{enumerate}
    \item \textbf{Noden vi skal fjerne har ingen barn.}\\ Vi kan da bare fjerne noden.
    \item \textbf{Noden vi skal fjerne har bare et barn.}\\ Vi kan da bare ta ut noden, og la barnet overta nodens far.
    \item \textbf{Noden vi skal fjerne har to barn.}\\ De to foregående tilfellene var enkle. Det er denne også, hvis man ser trikset. Spørsmålet vi må stille er: Hvilken annen node kan erstatte vår node, dvs. ta dens plass i treet? Det må være en med verdi større eller lik alle verdier i venstre subtre eller en med verdi mindre enn alle i høyre subtre. \newline \newline
    En løsning er å velge den noden med størst verdi i venstre subtre. Den finner du ved å gå til venstre barn og holde til høyre i hvert kryss videre nedover. Når du kommer til løvnoden har du funnet den du leter etter. \newline\newline
    En annen løsning er å velge den noden med minste verdi i høyre subtre. Helt tilsvarende går du da til høyre barn og holder til venstre i alle kryss helt til du når denne løvnoden.
    \newline\newline
    En av disse nodene kan klippes av og settes inn i stedet for den noden vi vil fjerne.
\end{enumerate}

\begin{figure}[H]
\includegraphics[scale=0.45]{images/fjernebarn}
\centering %centering the image
\caption{Fjerne barn}
\label{fig:fjernebarn}
\end{figure}

\subsection{Traversering}\\
Å traversere et tre eller en graf er et fint ord for å gå gjennom treet eller grafen. Her gir vi dere noen rekursive forklaringer på noen systematiske måter dette kan gjøres på.

\begin{itemize}
    \item \textbf{Prefiks:} Først utføres operasjonen på rotnoden, deretter prefikstraverseres venstre subtre (hvis det eksisterer) og til slutt prefikstraverseres høyre subtre (hvis det eksisterer). DFS. \textit{Kjører båt ved land, første man møter på.}
    \begin{figure}[H]
    \includegraphics[scale=0.45]{images/prefiks}
    \centering %centering the image
    \caption{Prefikstraversering}
    \label{fig:prefiks}
    \end{figure}
    \item \textbf{Infiks:} Først infikstraverseres venstre subtre, deretter utføres operasjonen på rotnoden og til slutt infikstraverseres høyre subtre. DFS. 
    \begin{figure}[H]
    \includegraphics[scale=0.45]{images/infiks}
    \centering %centering the image
    \caption{Infikstraversering}
    \label{fig:infiks}
    \end{figure}
    \item \textbf{Postfiks:} Først postfikstraverseres venstre subtre, deretter høyre og til slutt utføres operasjonen på rotnoden. DFS. \textit{Siste på hver gren merkes.}
    \begin{figure}[H]
    \includegraphics[scale=0.45]{images/postfiks}
    \centering %centering the image
    \caption{Postfikstraversering}
    \label{fig:postfiks}
    \end{figure}
    \item \textbf{Nivå:} Vi utfører først operasjonen på roten. Deretter tar vi for oss én og én rad på vei nediver, vanligvis fra venstre mot høyre. BFS. \textit{Merker fra venstre til høyre på hver rad.}
    \begin{figure}[H]
    \includegraphics[scale=0.45]{images/nivaa}
    \centering %centering the image
    \caption{Nivåtraversering}
    \label{fig:nivå}
    \end{figure}
\end{itemize}

\begin{boxed}
\begin{itemize}
    \item \textbf{Preorder}: Her printer man ut nodens verdi før dens barn, venstre og deretter høyre. 
    \begin{itemize}
        \item Eksempel fra figuren under: 10, 6, 3, 2, 1, 4, 5, 8, 7, 9, 13, 11, 12, 18, 15, 14, 16, 17
    \end{itemize}
    \item \textbf{Inorder}: Her printer man venstre barn, noden, og deretter høyre barn (om ikke det er noe venstre barn, print noden før høyre barn)
    \begin{itemize}
        \item Eksempel fra figuren under: 1, 2, 3, 4, 5, 6, 7, 8, 9, 10, 11, 12, 13, 14, 15, 16, 17, 18
    \end{itemize}
    \item \textbf{Postorder}: Her printer man nodens verdi etter man har printet venstre og høyre barn
    \begin{itemize}
        \item Eksempel fra figuren under: 1, 2, 5, 4, 3, 7, 9, 8, 6, 12, 11, 14, 17, 16, 15, 18, 13, 10
    \end{itemize}
\end{itemize}

\begin{figure}[H]
\includegraphics[scale=0.8]{images/tregraf}
\centering %centering the image
\caption{Tre}
\label{fig:tregraf}
\end{figure}
\end{boxed}

\subsection{Minimale spenntrær}
Som vi vet er et tre en forbundet graf uten sykler. Et \textbf{spenntre} er en subgraf av en graf \textit{G} som inneholder \textit{V - 1} kanter. Et \textbf{minimalt spenntre}, forkortet MST, av en vektet graf \textit{G} er et spenntre av \textit{G} som er slik at summen av kostnadene til kantene som inngår i treet, er minimal (det vil si at det ikke er mulig å lage andre spenntrær men mindre kantkostnader). Et MST er ikke nødvendigvis unikt – den kan altså finnes flere MST'er for samme graf.

\subsubsection{Prim}
Prinsippet med Prims algoritme er å alltid begynne med en tilfeldig node. I hvert steg ser man på alle kanter som forbinder en node som er med i treet man har bygget hittil med en node som ikke er med, og velger den kanten med minst kostnad. Dette gjøres ved å ha de ubrukte nodene i en prioritetskø, og markere hver node med den korteste kanten som forbinder den med en node i treet. Når alle nodene er med i treet, har du funnet et minimalt spenntre.
\\\\
Altså den mengden med kanter som minimerer summen av vekter samtidig som grafen er sammenhengende. Det minimale spenntreet reflekterer ikke nødvendigvis de faktisk korteste veiene fra en node til en annen.

\subsubsection{Kruskal}
Prinsippet med Kruskals algorimte er å sortere alle kantene etter kostnad, og se på kantene i stigende rekkefølge. Ta en kant (og nodene den går mellom) i spenntreet med mindre kanten ville ha laget en sykel. Fortsett til det ikke er flere kanter som kan legges til. Ideen er enkel, men for at algoritmen skal bli effektiv trenger man en avansert datastruktur som kalles \textit{disjoint set} for raskt å kunne finne ut om en kant vil danne en sykel eller ikke. Derfor anbefales Prim når man skal implementere en MST-algoritme selv.
\newpage
\section{Problemkompleksitet}
\subsection{P, NP, NPC}
\subsubsection{Eksempel på problem som er NPC}
\newpage
\section{Kjøretider til pensumalgoritmer}
\subsection{Sortering og velging}
\subsection{Heap-operasjoner}
\subsection{Graf-operasjoner}
\subsection{Andre algoritmer}

\end{document}