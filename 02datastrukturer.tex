\section{Datastrukturer}
En datastruktur er en måte å lagre og organisere data slik at man kan fasilitere og moderere de.

\subsection{Lenkede lister}
En lenket liste er en enkel datastruktur hvor objektene er ordnet lineært. Hvert objekt har helst to peker-variabler; \textit{next} og \textit{prev}. Har den begge disse peker-variablene er det en dobbelt lenket liste. Ved kun en peker (\textit{next}) er det en enkelt-lenket liste. Hvert element peker til det neste i rekkefølgen. Man kan også ha dobbeltlenkede lister, der hver node inneholder en verdi og peker til både forrige og neste node.
\\\\
Vi har forskjellige varianter av lenkede lister, totalt åtte kombinasjoner:
\begin{itemize}
    \item enkle eller doble
    \item sorterte eller usorterte
    \item sirkulære (siste og første element er koblet sammen) eller ikke-sirkulære
\end{itemize}

\begin{lstlisting}
    class Node:
    	def__init__(self):
    		self.value = None
    		self.next = None
    
    n1 = Node()
    n2 = Node()
    n3 = Node()
    n1.value = 1
    n2.value = 2
    n3.value = 3
    n1.next = n2
    n2.next = n3
\end{lstlisting}

\subsection{Abstrakte datastrukturer}
Forkortes ADT.

\subsubsection{Kø}
Kø er en \textbf{FIFO}; first in, first out. En kø implementeres som oftest som et array (gjerne et dynamisk array, likt Java sin ArrayList), men kan også implementeres som en lenket liste. En kø har egenskapene ENQUEUE og DEQUEUE. Det er ingen sniking i en kø, og elementer vil komme ut av køen i samme rekkefølge som de ble lagt inn. Elementer kan legges inn bakerst i køen til enhver tid og man kan når som helst hente ut det første elementet, med mindre køen er tom.

\begin{figure}[H]
\includegraphics[scale=0.7]{images/ko}
\centering %centering the image
\caption{Queue}
\label{fig:ko}
\end{figure}

\noindent En kø er en abstrakt datastruktur som bevarer de innsatte elementers rekkefølge. Enqueue er innsettingsoperasjonen, som setter inn et element på slutten av køen. Dequeue er uthentingsoperasjonen, den henter ut fra begynnelsen av køen.

\subsubsection{Stack}
Stack er en \textbf{LIFO}; last in, first out. Det kan sammenlignes med en stabel med tallerkener. Når en tallerken er nyvasket, legges den på toppen. Når du trenger en tar du den øverste i stabelen. Akkurat som at det ikke er lov å snike i en kø, er det ikke lov å stokke om på tallerkenene i en stack. Toppen av stacken er altså det eneste som er tilgjelgelig; du kan når som helst sette inn noe på toppen eller ta ut det øverste elementet (med mindre stacken er tom). En stack implementeres oftest med et array, eller en lenket liste. En stack har egenskapene INSERT, PUSH og POP (DELETE).

\begin{figure}[H]
\includegraphics[scale=0.7]{images/stack}
\centering %centering the image
\caption{Stack}
\label{fig:stack}
\end{figure}

\noindent En stakk er en abstrakt datastruktur som, i likhet med en kø, bevarer de innsatte elementers rekkefølge. Siden en stakk er LIFO både legger den til og tar ut fra slutten av strukturen.

\subsection{Heap}
Heaps er den datastrukturen flest har problem med å forstå, samtidig som at det kanskje er den mest nyttige. Styrken til en heap, sammenlignet med en liste, er at den er utrolig rask til å finne det største/minste elementet. Skal man f.eks. be en liste om å finne det minste elementet må man iterere gjennom hele listen, og derfor få en kjøretid på $(n)$. En heap klarer denne oppgaven med kjøretid $O(n log n)$. 
\\\\
En heap er en spesiell tresturktur, som tilfredsstiller heap-egenskapen. Det finnes ingen regler for hvordan søsken er ordnet i heapen. En heap er ofte implementert med et array. Det er viktig å legge merke til at første indeks skal være tom. Det vil si at i et 0-indeksert array vil plass 0 være tom, mens første node står på indeks 1.

\begin{table}[H]
    %\caption{}
    \label{tab:datastrukturer}
    \centering
    \begin{tabular}{|L{15em} | L{15em}|}
        \hline
        \rowcolor[HTML]{303F9F}
        \textbf{\textcolor{white}{Operasjon}} & \textbf{\textcolor{white}{ Tidskompleksitet}}\\
        \rowcolor[HTML]{E6E6E6}
        Finn min/max & $\theta(1)$\\
        Slett min/max & $\theta(log n)$\\
        \rowcolor[HTML]{E6E6E6}
        Sett inn & $\theta(log n)$\\
        Omstrukturere heap & $\theta(log n)$\\
         \rowcolor[HTML]{E6E6E6}
        Merge & $\theta(n)$\\
         \hline
    \end{tabular}
\end{table}

\noindent Måten en heap fungerer på er nesten litt ”magisk”. Ved å alltid la hver node i et binærtre ha en egenskap, så får man automagisk en heap. Denne egenskapen er: elementene som er direkte under noden har lavere verdi enn noden.

\begin{table}[H]
    %\caption{}
    \label{tab:heap}
    \centering
    \begin{tabular}{|L{10em} | L{10em}| L{10em}|L{10em}|}
        \hline
        \rowcolor[HTML]{303F9F}
        \textbf{\textcolor{white}{}} & \textbf{\textcolor{white}{Lenket liste}} & \textbf{\textcolor{white}{Array}} & \textbf{\textcolor{white}{Dynamisk array}}\\
        \rowcolor[HTML]{E6E6E6}
        Indeksering & $\theta(n)$ & $\theta(1)$ & $\theta(1)$\\
        Ins/del front & $\theta(1)$ & - & $\theta(n)$\\
        \rowcolor[HTML]{E6E6E6}
        Ins/del bakerst & $\theta(n)$ & - & $\theta(1)$\\
        Ins/del midten & $Søk + \theta(1)$ & - & $\theta(n)$\\
         \rowcolor[HTML]{E6E6E6}
        Unrukt plass & $\theta(n)$ & 0 & $\theta(n)$\\
         \hline
    \end{tabular}
\end{table}

\noindent I en heap må hele treet være fullstendig fylt ut, bortsett fra muligens det laveste nivået, som fylles ut fra venstre.
\begin{itemize}
    \item Rotnode: $i = 1$
    \item Foreldrenode(i): $i/2$
    \item Høyre-barn: $(2i + 1)$
    \item Venstre-barn: $(2i)$
\end{itemize}

\subsubsection{Max-heap}
For hver node $i$, bortsett fra rot-noden, må verdien til barnenodene være mindre eller lik foreldrenoden. $A[PARENT(i)]\geq A[i]$

\subsubsection{Min-heap}
For hver node $i$, bortsett fra rot-noden, må verdien til barnenodene være større eller lik foreldrenoden. $A[PARENT(i)]\leq A[i]$

\subsection{Chaining}

\subsection{Statiske datasett}
Kan ha worst-case $O(1)$ for søk.

\subsection{Amortisert analyse}

